
Este documento pretende servir como una guía para el estudio de algoritmos y estructuras de datos básicas para la composición de soluciones computacionales basadas en algoritmos sobre lenguajes imperativos. La estructura del documento corresponde en gran parte con el programa de estudios que ofrece la Licenciatura en Computación de la Universidad Central de Venezuela en su Plan de Estudio 2004, específicamente a la asignatura Algoritmos y Estructuras de Datos correspondiente al 2do período semestral de una carrera de 10 períodos. Así, se busca introducir al estudiante en las destrezas en el área de la algorítmica y la programación para la construcción de programas de manera sistemática y haciendo un uso eficiente de los recursos computacionales. Entonces se busca explicar conceptos teóricos involucrando a su vez el desarrollo de ejercicios prácticos.

El documento se divide en seis grandes partes: Tipos de Datos, Recursión, Backtracking, Complejidad, Estructuras Dinámicas, y Árboles. Para cada parte se trata de mostrar las siguientes secciones:
\begin{itemize}
\item Definiciones: Se muestran conceptos básicos dentro de un marco teórico asociado al tema a estudiar.
\item Clasificaciones/Implementaciones: Por cada sección se clasifica el tópico de estudio y/o se muestran las diversas implementaciones algorítmicas de las estructuras de datos o técnicas.
\item Ejercicios: Ciertos algoritmos/ejemplos se muestran para ser explicados en detalle.
\item Algoritmos: Un conjunto de algoritmos/ejemplos clásicos son mostrados explicando en qué consiste el problema.
\item Ideas Finales: Un conjunto de ideas principales para el cierre de cada sección.
\item Problemas: Se listan ejercicios simples para ser realizados por los estudiantes o en clases prácticas.
\end{itemize}

El orden de las secciones no es estricto, pero en muchas de ellas se requiere información previa. Por ejemplo, para el tema de Árboles se requiere del conocimiento base explicada en la sección Recursión por ser una estructura de datos recursiva por definición. Igualmente, es el facilitador/docente quién decide el orden al momento de difundir conocimiento.

Este documento no contiene todos los aspectos relacionados con cada sección, ya que trata de ser una guía para un período de clases de aproximadamente 4 horas semanales durante un período de 14 semanas (56 horas). Así, de antemano se recomienda complementar su utilización con otros textos o material bibliográfico de los tópicos explicados.

Todo el documento emplea la notación Alpha\footnote{La especificación completa de la notación Alpha está disponible en http://goo.gl/SOHo8h} la cual consiste en una notación algorítmica básica basada en pseudocódigo para la escritura de algoritmos y estructuras de datos. La idea es ser consistente al momento de la docencia y no incurrir en errores como mezclar notaciones de diversos lenguajes de programación. Además, la notación permite una rápida conversión a cualquier lenguaje de programación moderno.

Este documento puede contener errores. Por ello, se recomienda buscar la última versión en el repositorio git de dirección https://github.com/esmitt/notas-docencia-ayed.git. La ayuda siempre es bienvenida por ello puede enviar un \textit{pull request} en Github para colaborar.